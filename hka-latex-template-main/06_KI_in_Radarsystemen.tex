\part{KI in Radarsystemen}
\chapterimage{Bilder_und_Co/safety_sign_with_computer_code_and_a_airplane_in_the_background_1958542794.png} % Chapter heading image
%----------------------------------------------------------------------------------------
\chapter{KI Grundlagen}

\section{Begriffe}
ML vs. DL, überwachtes/teil-/unüberwachtes Lernen, Features vs. End-to-End.

\section{Datenpraxis}
Train/Valid/Test, saubere Splits (keine Leckagen), Domain Shift, Augmentation.

\section{Bewertung}
Precision/Recall/F1 pro Klasse, ROC/PR, Kalibrierung (ECE/Brier), Latenz/Throughput.

\section{Verlässlichkeit}
Unsicherheitsabschätzung, OOD-Erkennung, „Abstain/Unknown“, Erklärbarkeit (Saliency/Prototypen).

\section{Fehlermodi}
Overfitting, Datenbias, Verteilungsdrift, Spurious Correlations.




\chapter{KI im Radar}

\section{Einsatzfelder}
Clutter-Unterdrückung, Detektion, Objektklassifikation (z. B. Flugzeug/Heli/UAS/Vogel), 
Tracking/Track-Fusion, Ressourcen-Management (Waveform Scheduling).

\section{Eingaben}
RD/RDA-Tensors, Mikro-Doppler-Spektrogramme, Track-Sequenzen; 
physikgeleitete Features (RCS, Spektralbreite, Manöver).

\section{Modelle}
2D/3D-CNN/ViT (bildartig), LSTM/TCN/Transformer (zeitlich), hybride physik-informierte Netze.

\section{Runtime und Architektur}
Latenzbudgets, Quantisierung/Pruning für FPGA/SoC, Safety-Cage/Simplex als Fallback.

\section{Security-Basics}
Adversarial Examples (RD-Domäne), Data-Poisoning/Backdoors, Model-Stealing; 
harte vs. weiche Abwehrmaßnahmen.




\chapter{Wichtiges mitzunehmen}
\section{Merksatz}
In sicherheitskritischen Radaranwendungen ist KI kalibriert, OOD-fähig und in eine sichere Runtime-Architektur eingebettet.