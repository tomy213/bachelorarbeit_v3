\part{KI Sicherheit}
\chapterimage{Bilder_und_Co/a_radar_system_at_an_airport_tower_1403000730.png} % Chapter heading image
%----------------------------------------------------------------------------------------
\chapter{Begriffe} 

\section{AI Safety}
= beherrschbares, verlässliches Verhalten (funktionale Sicherheit, Fehlertoleranz).

\section{AI Security}
= Schutz der KI vor Angriffen (Poisoning, Adversarial, Model Theft).

\section{Airworthiness Security}
= luftfahrtspezifische Security-Prozesse.




\chapter{Risiken}
Safety-Fehler (Fehlklassifikation, schlechte Kalibrierung, OOD-Blindheit) 
vs. Security-Angriffe (Jamming/DRFM auf Sensor-Ebene, Backdoors/Adversarial auf ML-Ebene).

\chapter{Design-Prinzipien}
\section{Safety}
Kalibrierung, Unsicherheits-Schätzung, Abstain/Unknown, Safety-Cage/Simplex, erklärbare Features.

\section{Security}
Daten-Governance, signierte Modelle/secure boot, OOD/Anomalie-Wächter, adversarial Training, Red-Team-Tests.

\chapter{Metriken und Evidenz}
ECE/Brier, OOD-Rate, adversarial Erfolgsrate, J/S-Robustheit, Traceability (Req <-> Test <-> Befund).