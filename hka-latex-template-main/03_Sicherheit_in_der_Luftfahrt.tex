\part{Sicherheit in der Luftfahrt}
\chapterimage{Bilder_und_Co/a_radar_system_at_an_airport_tower_1403000730.png} % Chapter heading image
%----------------------------------------------------------------------------------------
\chapter{Sicherheit Grundlagen} 

\section{Safety vs. Security}
Unbeabsichtigte Fehler vs. absichtliche Angriffe.

\section{Risikobegriff}
Schweregrad × Eintrittswahrscheinlichkeit; fail-safe vs. fail-operational.

\section{Methoden}
Hazard-Analyse (FHA), FMEA/FMECA, Fault-Tree, Bow-Tie; „Defense in Depth“.

\section{Prozesse}
Anforderungen → Architektur → Verifikation/Validierung → Betrieb/Änderungsmanagement.



\chapter{Luftfahrt-spezifische Sicherheit} 

\section{Safety-Entwicklung}
ARP4754A (System), ARP4761 (Safety), DO-178C (Software), DO-254 (Hardware).

\section{Assurance Level}
DAL A–E je nach Kritikalität (z. B. Kollision vermeiden => hoch).

\section{Security-Framework}
DO-326A/356A/355A (Airworthiness Security, Methoden, Betrieb).

\section{Betrieb}
Continuing Airworthiness, Konfigurationskontrolle, Audits, Incident-Response.

\section{Metriken und Evidenz}
Nachweisbarkeit, Rückverfolgbarkeit (Requirements <-> Tests), Abdeckungsgrade.




\chapter{Wichtiges mitzunehmen}
\section{Merksatz}
In der Luftfahrt zählt nicht nur „funktioniert“, sondern „ist nachweislich sicher und beherrschbar“.