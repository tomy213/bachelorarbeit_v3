\part{Radar in der Luftfahrt}
\chapterimage{Bilder_und_Co/a_magnifying_glass_which_points_out_a_neuronal_network_surrounded_by_computer_code_and_a_question_ma_296098326.png} % Chapter heading image
%----------------------------------------------------------------------------------------
\chapter{Radar Grundlagen}

\section{Prinzip}
Senden–Empfangen von EM-Wellen, 
Laufzeit => Entfernung, 
Doppler => Relativgeschwindigkeit, 
Antenne/Beam => Winkel.

\section{Wichtige Größen}
SNR, PRF, Pulsbreite, 
Bandbreite/Entfernungsauflösung, 
Dopplerauflösung, Beamwidth, 
CFAR (Konstante Falschalarmrate).

\section{Datenformen}
I/Q-Daten → Range-Doppler-Map (RD) → 
ggf. Range-Doppler-Angle (RDA) → Detektionen/Tracks.

\section{Leistungskennzahlen}
Entdeckungswahrscheinlichkeit Pd, Falschalarmrate Pfa, 
Reichweite, Aktualisierungsrate, Latenz.

\section{Limits und Störer}
Clutter (Boden/Meer/Wetter), Mehrwege, RFI, 
Abschattung, Ambiguitäten (Range/Doppler).



\chapter{Radar im Luftfahrtkontext}

\section{Bodenradar}
Primärradar (PSR), 
Sekundärradar (SSR/Mode A/C/S), 
Multilateration/ADS-B (kooperativ).

\section{Bordradar}
Wetterradar (WXR), 
Kollisionsvermeidung/TAWS-Umfeld, 
Hinderniserkennung.

\section{Typischer Signalpfad}
Rohdaten → Pulskompression/FFT → CFAR → Clustering → Tracking → HMI.

\section{Leistungsanforderungen}
Abdeckung von Terminal/En-Route, Track-Kontinuität, 
Koexistenz mit ADS-B/TCAS, Verfügbarkeit/Redundanz.


\chapter{Wichtiges mitzunehmen}
\section{Merksatz}
Gute Klassifikation ist nur so gut wie die Stabilität von RD/RDA und der Track-Qualität.