\part{Grundlagen und Kontext}
\chapterimage{Bilder_und_Co/a_magnifying_glass_which_points_out_a_neuronal_network_surrounded_by_computer_code_and_a_question_ma_296098326.png} % Chapter heading image
%----------------------------------------------------------------------------------------
\chapter{Radar in der Luftfahrt}


\section{Radar Grundlagen}

\subsection{Prinzip}
Senden–Empfangen von EM-Wellen, 
Laufzeit => Entfernung, 
Doppler => Relativgeschwindigkeit, 
Antenne/Beam => Winkel.

\subsection{Wichtige Größen}
SNR, PRF, Pulsbreite, 
Bandbreite/Entfernungsauflösung, 
Dopplerauflösung, Beamwidth, 
CFAR (Konstante Falschalarmrate).

\subsection{Datenformen}
I/Q-Daten → Range-Doppler-Map (RD) → 
ggf. Range-Doppler-Angle (RDA) → Detektionen/Tracks.

\subsection{Leistungskennzahlen}
Entdeckungswahrscheinlichkeit Pd, Falschalarmrate Pfa, 
Reichweite, Aktualisierungsrate, Latenz.

\subsection{Limits und Störer}
Clutter (Boden/Meer/Wetter), Mehrwege, RFI, 
Abschattung, Ambiguitäten (Range/Doppler).



\section{Radar im Luftfahrtkontext}

\subsection{Bodenradar}
Primärradar (PSR), 
Sekundärradar (SSR/Mode A/C/S), 
Multilateration/ADS-B (kooperativ).

\subsection{Bordradar}
Wetterradar (WXR), 
Kollisionsvermeidung/TAWS-Umfeld, 
Hinderniserkennung.

\subsection{Typischer Signalpfad}
Rohdaten → Pulskompression/FFT → CFAR → Clustering → Tracking → HMI.

\subsection{Leistungsanforderungen}
Abdeckung von Terminal/En-Route, Track-Kontinuität, 
Koexistenz mit ADS-B/TCAS, Verfügbarkeit/Redundanz.


\section{Wichtiges mitzunehmen}
\subsection{Merksatz}
Gute Klassifikation ist nur so gut wie die Stabilität von RD/RDA und der Track-Qualität.








\chapter{Sicherheit in der Luftfahrt}
\section{Sicherheit Grundlagen} 

\subsection{Safety vs. Security}
Unbeabsichtigte Fehler vs. absichtliche Angriffe.

\subsection{Risikobegriff}
Schweregrad × Eintrittswahrscheinlichkeit; fail-safe vs. fail-operational.

\subsection{Methoden}
Hazard-Analyse (FHA), FMEA/FMECA, Fault-Tree, Bow-Tie; „Defense in Depth“.

\subsection{Prozesse}
Anforderungen → Architektur → Verifikation/Validierung → Betrieb/Änderungsmanagement.



\section{Luftfahrt-spezifische Sicherheit} 

\subsection{Safety-Entwicklung}
ARP4754A (System), ARP4761 (Safety), DO-178C (Software), DO-254 (Hardware).

\subsection{Assurance Level}
DAL A–E je nach Kritikalität (z. B. Kollision vermeiden => hoch).

\subsection{Security-Framework}
DO-326A/356A/355A (Airworthiness Security, Methoden, Betrieb).

\subsection{Betrieb}
Continuing Airworthiness, Konfigurationskontrolle, Audits, Incident-Response.

\subsection{Metriken und Evidenz}
Nachweisbarkeit, Rückverfolgbarkeit (Requirements <-> Tests), Abdeckungsgrade.




\section{Wichtiges mitzunehmen}
\subsection{Merksatz}
In der Luftfahrt zählt nicht nur „funktioniert“, sondern „ist nachweislich sicher und beherrschbar“.











\chapter{KI in der Luftfahrt}
\section{Use-Cases} 











\chapter{KI Sicherheit}
\section{Begriffe} 

\subsection{AI Safety}
= beherrschbares, verlässliches Verhalten (funktionale Sicherheit, Fehlertoleranz).

\subsection{AI Security}
= Schutz der KI vor Angriffen (Poisoning, Adversarial, Model Theft).

\subsection{Airworthiness Security}
= luftfahrtspezifische Security-Prozesse.




\section{Risiken}
Safety-Fehler (Fehlklassifikation, schlechte Kalibrierung, OOD-Blindheit) 
vs. Security-Angriffe (Jamming/DRFM auf Sensor-Ebene, Backdoors/Adversarial auf ML-Ebene).

\section{Design-Prinzipien}
\subsection{Safety}
Kalibrierung, Unsicherheits-Schätzung, Abstain/Unknown, Safety-Cage/Simplex, erklärbare Features.

\subsection{Security}
Daten-Governance, signierte Modelle/secure boot, OOD/Anomalie-Wächter, adversarial Training, Red-Team-Tests.

\section{Metriken und Evidenz}
ECE/Brier, OOD-Rate, adversarial Erfolgsrate, J/S-Robustheit, Traceability (Req <-> Test <-> Befund).











